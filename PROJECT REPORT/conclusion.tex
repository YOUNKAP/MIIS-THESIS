\chapter*{\normalsize{\textbf{CONCLUSION AND ADVANCEMENT}}}
%\addstarredchapter{CONCLUSION}
\addcontentsline{toc}{section}{\normalsize{\textbf{CONCLUSION AND ADVANCEMENT}}}
Meteorological satellite has proved to be very useful in weather analysis and forecasting.
Some of its applications are: 
\begin{itemize} 
\item Satellite imageries are used by the forecasters as first-hand information along with synoptic chart analysis while issuing the forecast.
\item Atmospheric sounding, clouds precipitation, convergence and divergence in the atmosphere, thunderstorm formation, various convective activities, can be derived / forecasted with the help of satellite pictures and their analysis. 
\item Cyclone formation, its intensity and movement (Track) can be monitored and predicted.
\end{itemize}
The advantages of meteorolgical satellite over ground observations  are :
\begin{itemize} 
\item Provides data of large region of the globe
\item Provides data of remote and inaccessible areas
\item Temporal data consistency 
\item Easy and rapid data collection
\item Relatively inexpensive
\item Rapid data interpretation
\end{itemize}
We also note some limitations of meteorolgical satellite as outlined below :
\begin{itemize} 
    \item Needs certain skill to interpret data
    \item Requires verification with ground observations
    \item Data from multiple sources may create confusion
    \item Objects can be misclassified
    \item Relative motion of sensor \& source may create 
    \item distortions in an image
\end{itemize}
