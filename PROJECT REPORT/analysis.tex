%\chapter{\normalsize{BACKGROUND OF SATELLITE REMOTE SENSING}}
\chapter{\normalsize{SATELLITE DATA ANALYSIS}}
\section{Technique 1: access data files}
\subsection{Access data files from NOAA CLASS}
\begin{itemize} 
\item The NOAA \href{https://www.avl.class.noaa.gov/saa/products/welcome}{Comprehensive Large Array-data Stewardship System (CLASS)} repository is the official site for accessing all available GOES-R Series Products.
\item  A similar but easier website to navigate is NCEI's Archive \href{https://www.ncdc.noaa.gov/airs-web/search}{Archive Information Request System (AIRS)}.
\end{itemize}
\subsection{Access data files from Amazon, Microsoft, OCC}
\begin{itemize}
\item \href{https://registry.opendata.aws/noaa-goes/}{Amazon Web Service (AWS)} - ABI L1b and L2+, GLM L2+, and SUVI L1b products are available in AWS S3 Buckets. These open datasets can be accessed by the public from AWS for free. 
\item \href{https://azure.microsoft.com/en-us/services/open-datasets/catalog/goes-16/}{Microsoft Azure}  - Two GOES-16 ABI Full Disk products are stored in a Azure blob container. 
The products currently available are L1b Radiances and L2+ MCMI, which can be accessed from Azure for free.
\item  \href{http://edc.occ-data.org/goes16/}{Open Commons Consortium (OCC)} - Stores a 100 TB rolling archive of GOES-16 data (\~{}8 months), the products stored are ABI L1b and ABI L2+ CMI and MCMI. OCC recommends using the AWS CLI or the python boto library, to access the data. 
\end{itemize}
\subsection{Access data files from Google Cloud}
\begin{itemize}
\item \href{https://console.cloud.google.com/marketplace/product/noaa-public/goes-16?filter=category:science-research&id=5babd633-afa0-4e40-9dba-0587f4aabc47}{Google Cloud} - ABI L1b and L2+, GLM L2+, and SUVI L1b products are available in two different buckets for GOES-16 and GOES-17. 
\item \href{https://developers.google.com/earth-engine}{Google Earth Engine (GEE)} - A cloud-based platform for geospatial analysis. This service runs through 
Google Cloud, and pulls GOES-R datasets from the Google Cloud buckets. 
\end{itemize}
\subsection{Use Python to retrieve data from AWS}
Python is the most powerful programming language using in data science and geospatial data analysis.
The following sample scripts demonstrate how to retrieve GOES-R files from AWS using  Python. 
\begin{lstlisting}[language=Python]
from goespy.Downloader import ABI_Downloader
### to use ABI_Downloader, you need 7 arguments:
import datetime as dt 
### Getting the current date (in UTC coordinate)
utcDateTime = dt.datetime.utcnow() 
## current year
year = utcDateTime.strftime("%Y")
# current month
month = utcDateTime.strftime("%m")
## current day
day = utcDateTime.strftime("%d")
## current hour in UTC 
hour = utcDateTime.strftime("%H")
##Choose a channel from your preference (can be C01-C16)
channel = ["C01"]
## In GOES satellite they have 9 products
## 3 are L1b-Rad(M,C,F)
## 3 are L2-CMIP(M,C,F)
## 3 are L2-MCMIP(M,C,F)
### In your case we will get the CMIPF, F means FullDisk (all the projection by the satellite)
product = 'L1b-L2-CMIPF'
## The Bucket is the variable contains the name of dataset server from goes on AWS
Bucket = 'noaa-goes16' ## in the future on AWS they will have goes17.
## Now we will call the function ABI_Downloader:
Abi = ABI_Downloader('noaa-goes16',year,month,day,hour,product,channel)
\end{lstlisting}
After all the dataset is downloaded, they are in your home directory with that structure:
\begin{figure}[H]
\begin{center}
\includegraphics[scale=0.8]{file1.png} %\cite{umhe}
\end{center}
\caption{L1b-L2-CMIPF product downloaded}
\label{L1b-L2-CMIPF product downloaded}%\cite{ABIA}
\end{figure}
\section{Technique 2: Explore satellite data and metadata}
\subsection{ GOES-R Series data format}
GOES-R Series product files use the netCDF-4 format.
NetCDF (Network Common Data Form) is a set of software libraries and machine-independent data formats that support the creation, access, and sharing of array-oriented scientific data. It is also a community standard for sharing scientific data.
\subsection{Read metadata}
Metadata provides information about the distinct items, such as: means of creation, purpose of 
the data, time and date of creation, creator or author of data, placement on a network (electronic 
form) where the data was created, what standards used etc.
The main purpose of metadata is to facilitate in the discovery of relevant information, more often 
classified as resource discovery. Metadata also helps organize electronic resources, provide 
digital identification, and helps support archiving and preservation of the resource.
Before any manipulation of data, it is vitaly important to read metadata.
Now let us see how to read the metadata of this satellite data file in command line using gdal tools.
\begin{lstlisting}[language=Bash]
## change directory to where you have downloaded the GLDAS data using the below
cd /mnt/d/MI_IS_DATA_ANALYSIS
## extatract metadata of the 'OR_ABI-L1b-RadM1-M3C02_G16_s20171931811268_e20171931811326_c20171931811356.nc' 
gdalinfo OR_ABI-L1b-RadM1-M3C02_G16_s20171931811268_e20171931811326_c20171931811356.nc
## press enter
\end{lstlisting}
\begin{figure}[H]
\begin{center}
\includegraphics[scale=0.5]{gdal1.png} %\cite{umhe}
\end{center}
\caption{metadata}
\label{metadta}%\cite{ABIA}
\end{figure}

\section{Technique 3: Plot a single band ABI Channel}
ABI imager has 16 channels, the code below show how to plot a single band ABI channel.
\begin{lstlisting}[language=Python]
def open_dataset(date, channel, idx, region):
    """
    Open and return a netCDF Dataset object for a given date, channel, and image index
    of GOES-16 data from THREDDS test server.
    """
    cat = TDSCatalog('https://thredds.ucar.edu/thredds/catalog/satellite/goes/east/products/'
                     f'CloudAndMoistureImagery/{region}/Channel{channel:02d}/{date:%Y%m%d}/catalog.xml')
    ds = cat.datasets[idx]
    ds = ds.remote_access(use_xarray=True)   
    return ds
def plot_GOES16_channel(date, idx, channel, region):
    """
    Get and plot a GOES 16 data band from the ABI.
    """
    ds = open_dataset(date, channel, idx, region)
    dat = ds.metpy.parse_cf('Sectorized_CMI')
    proj = dat.metpy.cartopy_crs
    x = dat['x']
    y = dat['y']
    fig = plt.figure(figsize=(10, 10))
    ax = fig.add_subplot(1, 1, 1, projection=proj)
    ax.add_feature(cfeature.COASTLINE, linewidth=2)
    ax.add_feature(cfeature.STATES, linestyle=':', edgecolor='black')
    ax.add_feature(cfeature.BORDERS, linewidth=2, edgecolor='black')
    for im in ax.images:
        im.remove()
    im = ax.imshow(dat, extent=(x.min(), x.max(), y.min(), y.max()), origin='upper')
    timestamp = datetime.strptime(ds.start_date_time, '%Y%j%H%M%S')
    add_timestamp(ax, time=timestamp, high_contrast=True, 
                  pretext=f'GOES 16 Ch.{channel} - ',
                  time_format='%d %B %Y %H%MZ', y=0.01,
                  fontsize=18)
    display(fig)
    plt.savefig("goes_fulldisk_C1.png")
    plt.close()
channel_list = {u'1 - Blue Band 0.47 \u03BCm': 1,
                u'2 - Red Band 0.64 \u03BCm': 2,
                u'3 - Veggie Band 0.86 \u03BCm': 3,
                u'4 - Cirrus Band 1.37 \u03BCm': 4,
                u'5 - Snow/Ice Band 1.6 \u03BCm': 5,
                u'6 - Cloud Particle Size Band 2.2 \u03BCm': 6,
                u'7 - Shortwave Window Band 3.9 \u03BCm': 7,
                u'8 - Upper-Level Tropo. WV Band 6.2 \u03BCm': 8,
                u'9 - Mid-Level Tropo. WV Band 6.9 \u03BCm': 9,
                u'10 - Low-Level WV Band 7.3 \u03BCm': 10,
                u'11 - Cloud-Top Phase Band 8.4 \u03BCm': 11,
                u'12 - Ozone Band 9.6 \u03BCm': 12,
                u'13 - Clean IR Longwave Band 10.3 \u03BCm': 13,
                u'14 - IR Longwave Band 11.2 \u03BCm': 14,
                u'15 - Dirty Longwave Band 12.3 \u03BCm': 15,
                u'16 - CO2 Longwave IR 13.3 \u03BCm': 16}
region = Select(
    options=['Mesoscale-1', 'Mesoscale-2', 'CONUS', 'PuertoRico', 'FullDisk'],
    description='Region:',)

channel = Dropdown(options=channel_list,value=16,description='Channel:',)

interact(plot_GOES16_channel, date=fixed(date), idx=fixed(-2), 
         channel=channel, region=region)

\end{lstlisting}

\begin{figure}[h!]
%\begin{figure}[H]
%\begin{figure}[t]
%\begin{figure}[b]
%\begin{figure}[b]
  \centering
  \begin{subfigure}[a]{0.4\linewidth}
    \includegraphics[width=\linewidth]{goes_fulldisk_C01.png}
     \caption{ABI CH01}
  \end{subfigure}
   %\centering
  \begin{subfigure}[b]{0.4\linewidth}
    \includegraphics[width=\linewidth]{goes_fulldisk_C05.png}
    \caption{ABI CH05}
  \end{subfigure}
   %\centering
  \begin{subfigure}[b]{0.4\linewidth}
    \includegraphics[width=\linewidth]{goes_fulldisk_C11.png}
    \caption{ABI CH11}
  \end{subfigure}
   %\centering
  \begin{subfigure}[b]{0.4\linewidth}
    \includegraphics[width=\linewidth]{goes_fulldisk_C16.png}
    \caption{ABI CH16}
  \end{subfigure}
  \caption{ABI BAND SAMPLE PLOT}
  \label{ABI BAND SAMPLE PLOT}
\end{figure}



\section{Convert NetCDF files to GeoTIFFs }
GeoTIFFS are regarded as the industry standard for georeferenced raster data.
The conversion of GOES-R netCDF arrays to single-band GeoTIFF rasters can be 
accomplished in command line. This conversion is very usefull because GeoTIFFS data format are easy to inject in Geographic Information System (GIS) software like QGIS.
\begin{lstlisting}[language=Bash]
##  Convert radiance from NetCDF files to GeoTIFF - below command
gdal_translate NETCDF:"OR_ABI-L1b-RadM1-M3C02_G16_s20171931811268_e20171931811326_c20171931811356.nc":OR_ABI-L1b-RadM1-M3C02_G16_s20171931811268_e20171931811326_c20171931811356.tif
\end{lstlisting}

\section{Technique 5: Processing satellite data using GIS Software}
\subsection{Objectives}
The main objective of this analysis is to process satellite data and compare with data obtained from at aeronautical met stations. Here we will use the three hour data from  \href{https://ldas.gsfc.nasa.gov/gldas}{GLDAS}.
\subsection{Avition meteorological stations in cameroon}


\begin{table}[H]
\caption{Aviation meteorological stations}
\label{tab:Aviation meteorological stations}
\begin{center}
\begin{tabular}{| l | c | c | c | c |}
\hline
\textbf{NAME} & \textbf{OMM CODE} & \textbf{ICAO CODE} & \textbf{LATITUDE} & \textbf{LONGITUDE}\\[2pt] \hline
Yaounde	& 64950&	FKYS&	03°50N&	011°31E \\\hline
Douala	& 64910	& FKKD	& 04°00N& 	009°44E \\\hline
Garoua	& 64860& 	FKKR& 	09°20N& 	013°23E\\\hline
Maroua	& 64851	& FKKL& 	10°27’N	& 14°15’E\\\hline
Ngaoundéré& 	64870& 	FKKN& 	07°21N& 013°34E\\\hline
Bafoussam& 	64894& 	FKKU& 	05°32’05 N& 	010°21’15 E\\\hline
Bertoua& 	64930& 	FKKO& 	04°36N& 	013°44E\\\hline
Bamenda& 	64892& 	FKKV& 	06°03N& 	010°07E\\\hline
\end{tabular}
\end{center}
\end{table}


\begin{figure}[H]
\begin{center}
\includegraphics[scale=0.8]{cmr_station.png} %\cite{umhe}
\end{center}
\caption{aviation meteorological stations}
\label{aviation meteorological stations}%\cite{ABIA}
\end{figure}
\subsection{GLDAS data}
The goal of the Global Land Data Assimilation System (GLDAS) is to ingest satellite- and ground-based observational data products, using advanced land surface modeling and data assimilation techniques, in order to generate optimal fields of land surface states and fluxes (Rodell et al., 2004a).
GLDAS data are available to download from this \href{https://hydro1.gesdisc.eosdis.nasa.gov/data/GLDAS/GLDAS_NOAH025_3H.2.1/}{link}.
Detailed documentation about GLDAS 2.1 product is available \href{https://hydro1.gesdisc.eosdis.nasa.gov/data/GLDAS/GLDAS_NOAH025_3H.2.1/doc/README_GLDAS2.pdf.}{here}  

\subsection{Specific requirements}
The requirements for the metereorological data at aeronautical station are:
\begin{figure}[H]
\begin{center}
\includegraphics[scale=0.8]{gldas1.png} %\cite{umhe}
\end{center}
\caption{meteorological data required}
\label{meteorological data required}%\cite{ABIA}
\end{figure}
The units at which GLDAS provide air temperature (K) and Pressure(Pa). Further GLDAS provide specific humidity. 
To match requirements the following conversion steps have to be performed:
\begin{description}
\item[Step 1] : Convert GLDAS air temperature from Kelvin to Deg C.
\item[Step 2] : Convert the unit of GLDAS pressure from Pa to Milli bar (Mb).
\item[Step 3] : Convert specific humidity to relative humidity following the description \href{https://earthscience.stackexchange.com/questions/2360/how-do-i-convert-specific-humidity-to-relative-humidity/}{here}.
\item[Step 4] : Extract the Ws, Tair, P and Rh corresponding at our aeronautical meteorological stations at observation time : 00:00, O3:00, O6:00, 09:00, 12:00, 15:00, 18:00 and 21:00 UTC.
\item[Step 5] : Compare the results with data collected at the stations.
\end{description}
\subsection{Download GLDAS data}
To download GLDAS data, we can use python script as shown below.
\begin{lstlisting}[language=Bash]
## In the Ubuntu session.
## Type in the following command to install the python package "gldas"
pip3 install gldas
## press enter
## Below code will download the GLDAS data between the dates provided in the command.
gldas_download -s 2021-01-01 -e 2021-01-31 --product GLDAS_Noah_v21_025 --username nina.younkap --password ****** /mnt/path/to/folder
## Note that the last argument in above command is path to a folder where the downloaded files will be stored.
\end{lstlisting}
\subsection{Processing single data file}
Let us now see how to process a single NetCDF file (.nc4) dowloaded from GLDAS using the previous step and perform the required conversions.
For example let us consider the GLDAS data representing 06:00 hours on 31 January 2021.
The GLDAS file name follows a particular structure.
\newline
The file name is \textbf{GLDAS\_NOAH025\_3H.A20210131.0600.021.nc4}
\begin{itemize}
\item \textcolor{magenta}{GLDAS\_NOAH025\_3H} represents the name, spatial resolution (025 means 0.25 degree resolution) and temporal resolution (3H means three hour data)
\item \textcolor{magenta}{A20210131}  represents the date of acquisition
\item \textcolor{magenta}{0600}  represents the time (in this case 06:00 GMT) the parameters are computed for.
\item \textcolor{magenta}{021}  represents the version of GLDAS data, in this case 2.1.
\item \textcolor{magenta}{nc4}  represents the extension and format of the data, in this case netCDF4.
\end{itemize}
Now let us see how to read the metadata of this file in command line using gdal tools
\begin{lstlisting}[language=Bash]
## change directory to where you have downloaded the GLDAS data using the below
cd /mnt/d/mi_is_project_data/2021/31"
## Extract metadata of the 'GLDAS_NOAH025_3H.A20210131.0600.021.nc4' in the file "metadata.txt"
gdalinfo GLDAS_NOAH025_3H.A20210131.0600.021.nc4 > metadata.txt
## press enter
## Read metadata using vim editor
vim metadata.txt
\end{lstlisting}
In the metadata, under Subdatasets: all the parameters provided as subdatasets are listed. 
These subdataset names are used in the further steps to process individual parameters. For example, NETCDF:\textcolor{magenta}{"GLDAS\_NOAH025\_3H.A20210131.0600.021.nc4":Tair\_f\_inst} is the name of air temperature grid at 06:00 on 31 January 2021 and we will use this name to extract/process this single grid. 
The codes of the parameters are listed in the \href{https://hydro1.gesdisc.eosdis.nasa.gov/data/GLDAS/GLDAS_NOAH025_3H.2.1/doc/README_GLDAS2.pdf}{GLDAS user manual} (Table 3.1).
\newline
So for our project we need the following subdatasets from a single GLDAS file:
\newline
\vspace{0.5\baselineskip}
\textcolor{magenta}{NETCDF:"GLDAS\_NOAH025\_3H.A20210131.0600.021.nc4":Qair\_f\_inst} - Specific humidity (Kg/Kg)
\newline
\vspace{0.5\baselineskip}
\textcolor{magenta}{NETCDF:"GLDAS\_NOAH025\_3H.A20210131.0600.021.nc4":Psurf\_f\_inst} - Surface Pressure (Pa)
\newline
\vspace{0.5\baselineskip}
\textcolor{magenta}{NETCDF:"GLDAS\_NOAH025\_3H.A20210131.0600.021.nc4":Tair\_f\_inst} - Air Temperature (K)
\newline
\vspace{0.5\baselineskip}
\textcolor{magenta}{NETCDF:"GLDAS\_NOAH025\_3H.A20210131.0600.021.nc4":Wind\_f\_inst } - Wind speed (m/s)
\newline
\vspace{0.5\baselineskip}
\textcolor{magenta}{NETCDF:"GLDAS\_NOAH025\_3H.A20210131.0600.021.nc4":SWdown\_f\_inst} - Downward short-wave radiation flux (W m-2)
\newline
Now let us extract each of the above subdatasets and convert them into tif file. For this we will use a gdal command called \textcolor{magenta}{gdal\_translate} which is a global conversion tool for all kind of raster formats. Detailed documentation on gdal\_translate is given \href{https://gdal.org/programs/gdal_translate.html}{here}.
\begin{lstlisting}[language=Bash]
##  Convert specific humidity to tif - below command
gdal_translate NETCDF:"GLDAS_NOAH025_3H.A20210131.0600.021.nc4":Qair_f_inst GLDAS_NOAH025_3H_A20210131_0600_Qair.tif
## Convert Surface Pressure to tif - below command
gdal_translate NETCDF:"GLDAS_NOAH025_3H.A20210131.0600.021.nc4":Psurf_f_inst GLDAS_NOAH025_3H_A20210131_0600_Psurf.tif
## Convert air temperature to tif - below command
gdal_translate NETCDF:"GLDAS_NOAH025_3H.A20210131.0600.021.nc4":Tair_f_inst GLDAS_NOAH025_3H_A20210131_0600_Tair.tif
## Convert Wind speed to tif - below command
gdal_translate NETCDF:"GLDAS_NOAH025_3H.A20210131.0600.021.nc4":Wind_f_inst GLDAS_NOAH025_3H_A20210131_0600_Wind.tif
## Convert Short wave downward radiation to tif - below command
gdal_translate NETCDF:"GLDAS_NOAH025_3H.A20210131.0600.021.nc4":SWdown_f_tavg GLDAS_NOAH025_3H_A20210131_0600_SWdown.tif
## Press enter
\end{lstlisting}
Now that we have the required parameters of 31 January 2021 in tif format, let us open the air temperature map in QGIS and see how it looks!
\begin{figure}[H]
\begin{center}
\includegraphics[scale=0.6]{qgis1.png} %\cite{umhe}
\end{center}
\caption{GLDAS Tair data displayed in QGIS}
\label{GLDAS Tair data displayed in QGIS}%\cite{ABIA}
\end{figure}
\paragraph{}
Let us also zoom to Cameroon boundaries and have a look at the Tair map.
Remember the resolution of data is 0.25 degrees.
\begin{figure}[H]
\begin{center}
\includegraphics[scale=0.6]{qgis2.png} %\cite{umhe}
\end{center}
\caption{GLDAS Tair data displayed in QGIS and zoomed to Cameroon}
\label{GLDAS Tair data displayed in QGIS and zoomed to Cameroon}%\cite{ABIA}
\end{figure}
Now let us see how to do unit conversion of all the five parameters required for aviation. 
One additional step is to clip the unit converted maps to cameroon boundaries as we are only interested in that region not global.
\newline
For further steps let us move to \textcolor{magenta}{GRASS GIS} as it is easier to do spatial and temporal analysis with GRASS library. 
We will create a new location and mapset for processing GLDAS data for Cameroon.
\begin{lstlisting}[language=Bash]
# Create (-c) just the location called "cmr" in epsg:4326 and exit (-e)
grass78 epsg:4326 /mnt/d/grassdata/cmr -c -e
# Create (-c) mapset called "met_data" inside the location "cmr" and open GRASS GIS in "cmr/met_data" mapset
grass78 /mnt/d/grassdata/cmr/met_data -c
# "-c" flag in above command is required only one time to create the mapset met_data
# Afterwards use below command to just start the existing mapset.
# grass78 /mnt/d/grassdata/cmr/met_data
\end{lstlisting}
Import the cameroon vector file , aviation meteorological station file , and others file required for our project into grass location. 
\begin{lstlisting}[language=Bash]
## IMPORT VECTOR DATA: Boundaries of Cameroon an  Aviation meteorological station in cameroon
## Navigate (cd) to the folder Cmr_Base_Layers
cd /mnt/d/mi_is_project_data/Cmr_Base_Layers 
# Import 'Cameroon' boundary shapefile into a vector in Grass GIS
v.import in=cmr.shp out=cmr
# Import 'Aviation meteorological staitions' points shapefile into a vector in Grass GIS
v.import in=aero_met_station_cmr.shp out=met_station
# Import 'FKKR TMA2' boundary shapefile into a vector in Grass GIS
v.import in=Fkkr_tma2.shp out=fkkr_tma2
# Import 'FKKN TMA' boundary shapefile into a vector in Grass GIS
v.import in=Fkkn_tma.shp out=fkkn_tma
# Import 'FKKL CONTROL ZONE' boundary shapefile into a vector in Grass GIS
v.import in=Fkkl_ctr.shp out=fkkl_ctr
# set the computational region to Cameroon Boundaries and set the computational resolution to 0.25 degrees
g.region vector=cmr res=0.25 -a
\end{lstlisting}
To automate the workflows, the following bash script process all the 5 required parameters (Tair, Rh, Wind speed, Psurf and SWDown), clipped on cameroon boundaries  and converted to the required units from .nc4 files in a day.
\begin{lstlisting}[language=Bash]
#!/bin/bash
## This script process a single day GLDAS data and do all the required conversions needed for our project.
## GENERAL ##
if [ -z "$GISBASE" ] ; then
    echo "You must be in GRASS GIS to run this program." >&2
    exit 1
fi
# Set a environment to enable overwrite by default
export GRASS_OVERWRITE=1

# Navigate to the folder containing single day .nc4 files
#e.g INDAT="/mnt/d/mi_is_project_data/Cmr_Base_Layers/2021/01" for 01 January 2021
#Here we work on the  31 January 2021 data
INDAT="/mnt/d/mi_is_project_data/2021/031" 
cd ${INDAT}
# set the computational region to Urmia Lake basin and set the computational resolution to 0.25 degrees
g.region vector=cmr res=0.25 -a
# set the mask to cameroon boundaries
#r.mask vect=cmr

# For loop to process all the .nc files in one go
for i in `ls GLDAS*.nc4`; do
    dt=`echo ${i}|cut -d. -f2-3`
    #  Convert specific humidity to tif - below command
    gdal_translate NETCDF:"${i}":Qair_f_inst GLDAS_NOAH025_3H_${dt}_Qair.tif
    # Convert Surface Pressure to tif - below command
    gdal_translate NETCDF:"${i}":Psurf_f_inst GLDAS_NOAH025_3H_${dt}_Psurf.tif
    # Convert air temperature to tif - below command
    gdal_translate NETCDF:"${i}":Tair_f_inst GLDAS_NOAH025_3H_${dt}_Tair.tif
    # Convert Wind speed to tif - below command
    gdal_translate NETCDF:"${i}":Wind_f_inst GLDAS_NOAH025_3H_${dt}_Wind.tif
    # Convert Short wave downward radiation to tif - below command
    gdal_translate NETCDF:"${i}":SWdown_f_tavg GLDAS_NOAH025_3H_${dt}_SWdown.tif
    # Import to GRASS
    # Import and clip specific humidity
    r.import in=GLDAS_NOAH025_3H_${dt}_Qair.tif out=GLDAS_NOAH025_3H_${dt}_Qair -o
    # Import and clip Surface Pressure
    r.import in=GLDAS_NOAH025_3H_${dt}_Psurf.tif out=GLDAS_NOAH025_3H_${dt}_Psurf -o
    # Import and clip Tair
    r.import in=GLDAS_NOAH025_3H_${dt}_Tair.tif out=GLDAS_NOAH025_3H_${dt}_Tair -o
    # Import and clip Wind speed
    r.import in=GLDAS_NOAH025_3H_${dt}_Wind.tif out=GLDAS_NOAH025_3H_${dt}_Wind -o
    # Import and clip Short wave radiation
    r.import in=GLDAS_NOAH025_3H_${dt}_SWdown.tif out=GLDAS_NOAH025_3H_${dt}_SWdown -o
    # Unit conversion
    # Air temperature from Kelvin to degree celsius
    r.mapcalc "GLDAS_NOAH025_3H_${dt}_Tair_final = GLDAS_NOAH025_3H_${dt}_Tair - 273.15"
    # Short wave radiation (no conversion required)
    r.mapcalc "GLDAS_NOAH025_3H_${dt}_SWdown_final = GLDAS_NOAH025_3H_${dt}_SWdown"
    # Wind speed (no conversion required)
    r.mapcalc "GLDAS_NOAH025_3H_${dt}_Wind_final = GLDAS_NOAH025_3H_${dt}_Wind"
    ## Specific humidity re saved
    r.mapcalc "GLDAS_NOAH025_3H_${dt}_Qair_final = GLDAS_NOAH025_3H_${dt}_Qair"
    ## Pressure convert from pa to mb
    r.mapcalc "GLDAS_NOAH025_3H_${dt}_Psurf_final = GLDAS_NOAH025_3H_${dt}_Psurf / 100"
    ## Humidity according to the url: https://earthscience.stackexchange.com/questions/2360/how-do-i-convert-specific-humidity-to-relative-humidity
    r.mapcalc "es = 6.112 * exp((17.67 * GLDAS_NOAH025_3H_${dt}_Tair_final) / (GLDAS_NOAH025_3H_${dt}_Tair_final + 243.5))"
    r.mapcalc "e = (GLDAS_NOAH025_3H_${dt}_Qair_final * GLDAS_NOAH025_3H_${dt}_Psurf_final) / (0.378 * GLDAS_NOAH025_3H_${dt}_Qair_final + 0.622)"
    r.mapcalc "GLDAS_NOAH025_3H_${dt}_Rh = (e / es) * 100"
    # Final Relative humidity in %
    r.mapcalc "GLDAS_NOAH025_3H_${dt}_Rh_final = float(if(GLDAS_NOAH025_3H_${dt}_Rh > 100, 100, if(GLDAS_NOAH025_3H_${dt}_Rh < 0, 0, GLDAS_NOAH025_3H_${dt}_Rh)))"

done
# Remove the mask
#r.mask -r
\end{lstlisting}
We save the script above in the file \textit{myscript.sh}. The following script show how to run this file on Linux Operating System.
\begin{lstlisting}[language=Bash]
# Run the following command to install dos2unix
sudo apt-get install dos2unix
# Below command removes the trailing spaces from your script
dos2unix myscript.sh
# Run the following command to run the above saved script file
sh myscript.sh
# press enter
\end{lstlisting}
\subsection{Results}
\subsection{Temperature chart}
\begin{lstlisting}[language=Bash]
#dt="20210131.0300"
dt="20210131.1500"
# Visualize data
r.mask vect=cmr --o
# Open a monitor
d.mon wx0
# Display a raster map
d.rast GLDAS_NOAH025_3H_A${dt}_Tair_final
# Display a vector map
d.vect map=cmr type=boundary
d.text text="31 JAN 2021 0300Z" color=black bgcolor=white size=3
# Add raster legend
d.legend -t -s -b raster=GLDAS_NOAH025_3H_A${dt}_Tair_final title=TEMPERATURE title_fontsize=20 font=sans fontsize=18
# Add North arrow
d.northarrow style=1b text_color=black
#Clear the monitor
#d.erase -f
#r.mask -r
# press enter
\end{lstlisting}

\begin{figure}[H]
\begin{center}
\includegraphics[scale=0.8]{tp03.png} %\cite{umhe}
\end{center}
\caption{Temperature at 03:00 UTC}
\label{Temperature at 03:00 UTC}%\cite{ABIA}
\end{figure}

\begin{figure}[H]
\begin{center}
\includegraphics[scale=0.8]{tp15.png} %\cite{umhe}
\end{center}
\caption{Temperature at 15:00 UTC}
\label{Temperature at 15:00 UTC}%\cite{ABIA}
\end{figure}

\subsection{Relative Humidity Chart}
\begin{lstlisting}[language=Bash]
dt="20210131.0300"
#dt="20210131.1500"
# Visualize data
r.mask vect=cmr --o
# Open a monitor
d.mon wx0
# Display a raster map
d.rast GLDAS_NOAH025_3H_A${dt}_Rh_final
# Display a vector map
d.vect map=cmr type=boundary
d.text text="31 JAN 2021 0300Z" color=black bgcolor=white size=3
# Add raster legend
d.legend -t -s -b raster=GLDAS_NOAH025_3H_A${dt}_Rh_final title=RELATIVE HUMIDITY title_fontsize=20 font=sans fontsize=18
# Add North arrow
d.northarrow style=1b text_color=black
#Clear the monitor
#d.erase -f
#r.mask -r
\end{lstlisting}

\begin{figure}[H]
\begin{center}
\includegraphics[scale=0.6]{rh03.png} %\cite{umhe}
\end{center}
\caption{Relative Humidity at 03:00 UTC}
\label{Relative Humidity at 03:00 UTC}%\cite{ABIA}
\end{figure}

\begin{figure}[H]
\begin{center}
\includegraphics[scale=0.6]{rh15.png} %\cite{umhe}
\end{center}
\caption{Relative Humidity at 15:00 UTC}
\label{Relative Humidity at 15:00 UTC}%\cite{ABIA}
\end{figure}

\subsection{Surface Pressure Chart}
\begin{lstlisting}[language=Bash]
#dt="20210131.0300"
dt="20210131.1500"
# Visualize data
r.mask vect=cmr --o
# Open a monitor
d.mon wx0
# Display a raster map
d.rast GLDAS_NOAH025_3H_A${dt}_Psurf_final 
# Display a vector map
d.vect map=cmr type=boundary
d.text text="31 JAN 2021 1500Z" color=black bgcolor=white size=3
# Add raster legend
d.legend -t -s -b raster=GLDAS_NOAH025_3H_A${dt}_Psurf_final title="SURFACE PRESSURE (Pa)" title_fontsize=20 font=sans fontsize=18
# Add North arrow
d.northarrow style=1b text_color=black
#d.erase -f
#r.mask -r
\end{lstlisting}

\begin{figure}[H]
\begin{center}
\includegraphics[scale=0.6]{sp03.png} %\cite{umhe}
\end{center}
\caption{Surface Pressure at 03:00 UTC}
\label{Surface Pressure  at 03:00 UTC}%\cite{ABIA}
\end{figure}

\begin{figure}[H]
\begin{center}
\includegraphics[scale=0.6]{sp15.png} %\cite{umhe}
\end{center}
\caption{Surface Pressure  at 15:00 UTC}
\label{Surface Pressure  at 15:00 UTC}%\cite{ABIA}
\end{figure}


\subsection{Wind Speed Chart}
\begin{lstlisting}[language=Bash]
#dt="20210131.0300"
dt="20210131.1500"
# Visualize data
r.mask vect=cmr --o
# Open a monitor
d.mon wx0
# Display a raster map
d.rast GLDAS_NOAH025_3H_A${dt}_Wind_final
# Display a vector map
d.vect map=cmr type=boundary
d.text text="31 JAN 2021 1500Z" color=black bgcolor=white size=3
# Add raster legend
d.legend -t -s -b raster=GLDAS_NOAH025_3H_A${dt}_Wind_final title="WIND SPEED m/s" title_fontsize=20 font=sans fontsize=18
#Add North arrowd.northarrow style=1b text_color=black
d.northarrow style=1b text_color=black
#d.erase -f
#r.mask -r

\end{lstlisting}

\begin{figure}[H]
\begin{center}
\includegraphics[scale=0.6]{ws03.png} %\cite{umhe}
\end{center}
\caption{Wind Speed at 03:00 UTC}
\label{Surface Pressure  at 03:00 UTC}%\cite{ABIA}
\end{figure}

\begin{figure}[H]
\begin{center}
\includegraphics[scale=0.6]{ws15.png} %\cite{umhe}
\end{center}
\caption{Wind Speed at 15:00 UTC}
\label{Surface Pressure  at 15:00 UTC}%\cite{ABIA}
\end{figure}




