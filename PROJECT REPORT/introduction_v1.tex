\chapter*{\normalsize{\textbf{INTRODUCTION}}}
%\addstarredchapter{INTRODUCTION}
\addcontentsline{toc}{section}{\normalsize{\textbf{INTRODUCTION}}}
\paragraph{}
Weather monitoring and forecasting was one of the first civilian applications of satellite 
remote sensing, dating back to the first true weather satellite, TIROS-1 launched in 1960 by the United States. %\newline
Several other weather satellites were launched over the next five years, in near-polar orbits, providing repetitive 
coverage of global weather patterns. %\newline
In 1966,  National Aeronautics and Space Administration (NASA) launched the geostationary Applications Technology 
Satellite (ATS-1) which provided hemispheric images of the Earth's surface and cloud cover every half hour. 
For the first time, the development and movement of weather systems could be routinely monitored. %\newline
Today, several countries (India, China, Russia, Japan) operate weather, or 
meteorological satellites to monitor weather conditions around the globe. %\newline
Generally speaking, these satellites use sensors which have fairly coarse spatial resolution (when compared to 
systems for observing land) and provide large areal coverage. 
Their temporal resolutions are generally quite high, providing frequent observations of the 
Earth's surface, atmospheric moisture, and cloud cover, which allows for near continuous 
monitoring of global weather conditions, and hence  forecasting.\newline
The current project report \textbf{ Meteorological  satellite : Sensing principles, data collection  and analysis} is structured as outlined below.\newline
The first chapter is a brief overview of processes and steps involve in satellite remote sensing. \newline
The second chapter is a case study, we choose GOES-R  series satellites, which cover East Africa countries and is used by meteorologists for weather monitoring and forecasting. \newline
The third chapter deals about different ways to access GOES-R series data and visualise a single band image. \newline
The fourth chapter covers satellite data analysis, we will process three hours satellite data from GLDAS and compare with data obtained
from direct observation at aviation meteorolgical stations. \newline
The fith chapter presents the results and interpretation that we can derive from our analysis.\newline
Satellite data  are large amount of data, so analysis  requires computer hardware with good capabilities, softwares and programming skills. \newline
For completion of the current work, we use the following tools: \newline
\textbf {Computer harward}: HP Probook, Intel R Core i5, RAM 16 GB. \newline
\textbf {Operating systems} : Ubuntu 20.04.4 \newline
\textbf {Processing} : GRASS GISS 7.8  with bash script for worklow automation. \newline
\textbf {Visualisation} : python 3.8 and  QGIS 16 LTR.


