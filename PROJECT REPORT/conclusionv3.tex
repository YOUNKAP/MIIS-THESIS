\chapter*{\normalsize{\textbf{CONCLUSION AND FUTURE ENHANCEMENT}}}
%\addstarredchapter{CONCLUSION AND FUTURE ENHANCEMENT}
\addcontentsline{toc}{section}{\normalsize{\textbf{CONCLUSION AND FUTURE ENHANCEMENT}}}
\paragraph{}
This project enable us to succesfully perform spatio-temporal analysis of satellite data through the following operations:
\begin{itemize} 
    \item Extraction of single meteorological parameters (temperature, relative humidity, surface pressure and  wind) required for avaition safety and efficiency;
    \item Conversion of temperature from Kelvin to Deg C and   surface pressure from Pa to hpa;
    \item Conversion from specific humidity to relative humidity ;
    \item Computation and visualisation on thematic  chart  of meteorological parameters  to the extent of Cameroon boundaries ;
    \item Extraction of meteorological parameters  corresponding to our ground observation points;
    \item Comparison between data extracted from satellite remote sensing and those directly obtained in situ at ground observation points. 
\end{itemize}
\paragraph{}
We found that on the whole the satellite \textbf {over estimates} temperature and \textbf {under estimates} pressure and relative humidity. \newline
But due to the time allow for this project  the comparison between satellite data and in situ data was limited to two stations (Ngaoundere and Garoua) and on a single day (31 January 2021). \newline
These space  and time  limitation can not enable us to draw a global picture representative for the whole cameroon boundaries.
So, as enhancement for this work, we are planing to:
\begin{itemize} 
    \item Extend our study space on  the 19 meteorological stations (8 aviation and 11 agro) that compose  the cameroon meteorolgical network ;
    \item Extend our study time on the last 5, 10 and 30 years ;
    \item Extend our study to others meteorological parameters required for agriculture (AETI, Net radiation, precipittation,  NDVI , transpiration, PET, etc.).
\end{itemize}
\paragraph{}
Given the fact that the installation, operation and maintenance of conventional or automatic weather station is expensive and unaffordable
for many farmers. We aim to provide farmers with \textbf {free} weather data derived from publicly available satellites (Goes-R, Landsat, EumetSat, Modis etc.) 
to fufill their meteorological needs and support their business operations.
%and also the issues related to climate change.