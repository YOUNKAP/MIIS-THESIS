\chapter*{\normalsize{\textbf{CONCLUSION AND ADVANCEMENT}}}
%\addstarredchapter{CONCLUSION}
\addcontentsline{toc}{section}{\normalsize{\textbf{CONCLUSION AND ADVANCEMENT}}}
During the current work, we have done  spatio-temporal analysis of satellite data and successfully compltete the following operations:
\begin{itemize} 
    \item Download and plot a single band ABI image from GOESR – Series satellite;
    \item Download satellite derived data from GLDAS;
    \item Metadata discovery and extraction ;
    \item Extraction of single meteorological parameters (temperature, relative humidity, pressure and surface wind) required for avaition safety and efficiency;
    \item Conversion of temperature from Kelvin to Deg C and   surface pressure from Pa to hpa;
    \item Conversion from specific humidity to relative humidity ;
    \item Computation and visualisation on thematic  chart  of meteorological parameters  to the extent of Cameroon boundaries ;
    \item Extraction of meteorological parameters  corresponding to our ground observation points (aviation meteorological station);
    \item Comparison between data extracted from satellite remote sensing and those directly obtained in situ at ground observation points (aviation meteorological station). 
\end{itemize}
The results of our analysis show that on the whole the satellite \textbf {over estimates} the temperature and \textbf {under estimates} pressure and relative humidity. \newline
But due to the time allow for this project  the comparison between satellite data and in situ data was limited to two stations (Ngaoundere and Garoua) on a single day (31 January 2021). \newline
These space  and time  limitations can not enable us to draws a global picture representative for the whole cameroon boundaries.
So, as enhancement for this work, we are planing to:
\begin{itemize} 
    \item extend our study space on  the 19 meteorological stations (8 aviation and 11 agro) that compose  the cameroon meteorolgical network ;
    \item extend our study time on the last 5, 10 and 30 years ;
    \item extend our study to others meteorological parameters required for agriculture (AETI, Net radiation, precipittation,  NDVI , transpiration,PET, etc.) ;
\end{itemize}
Given the fact that the installation, operation and maintenance of a conventional or automatic weather station is expensive and unaffordable
for many farmres. We aim to provide farmers with free weather data derived from publicly available satellites (Goes-R, Landsat, EumetSat, Modis etc.) 
to fufill their meteorological needs and support their business.
%and also the issues related to climate change.